% !TEX root = main.tex

\section{Grundlagen}

\subsection{Wellengleichung}\label{sec:wave_equation}
%TODO: Überarbeiten --> zweite Gleichung rausnehmen und als Text schreiben
Die Basis der zu entwickelnden Programme bildet folgende Gleichung:
\begin{quote}
	$A(i, t + 1) = 2A(i, t) - A(i, t - 1) + c(A(i - 1, t) - 2A(i, t) + A(i + 1, t))$
\end{quote}
Dabei ist $t$ der Zeitpunkt Schwingung, $i$ der Index der Welle und A(i,t) steht für die Amplitude zum Zeitpunkt t und am Punkt i.\\
Klarer wird diese wenn man sich A(i, t+1), also das Ergebnis der Gleichung, als den zukünftigen Wert an der aktuellen Stelle (futureVal) vorstellt. Dann würde die Gleichung ungefähr so aussehen:
\begin{quote}
	$futureVal = 2 * presentVal - pastVal + c * (presentLeftNeighbourVal - 2 * presentVal + presentRightNeighbourVal)$
\end{quote} 
Die Variable C ist eine Konstante, welche der Schrittweite zwischen zwei benachbarten Zeitpunkten t entspricht. Damit kann die Geschwindigkeit der Wellenbewegung beeinflusst werden.

\subsection{Parallelisierung}
Kurz nach der Jahrtausendwende sank die Leistungssteigerung, die bei Einkernprozessoren jedes Jahr erreicht wurde stark ab. Eine Ursache hierfür ist die sog. Power-Wall. Wenn die Taktfrequenz eines Prozessors erhöht wird, muss auch die anliegende Spannung erhöht werden. Dies führt aber dazu, dass mehr Wärme erzeugt wird, welche abgeführt werden muss. Die Power-Wall ist heute im Prinzip schon erreicht. Wenn man jedoch noch mehr Leistung haben will, können Mehrkernprozessoren Abhilfe schaffen.\\
Ein Großteil der Parallelisierung findet auf der Prozess- oder auf der Threadebene statt. Hier sind große Performance-Verbesserungen möglich. Diese benötigen aber spezielle Anpassungen des Quellcodes. Um diese Anpassungen geringer zu halten stehen viele verschiedene Frameworks zur Verfügung. Zwei davon werden in dieser Arbeit genauer betrachtet.

\subsection{OpenMP}
Auf einem Computer (PC) befindet sich in der Regel ein eng gekoppeltes Speichersystem, d.h. alle Prozessoren greifen auf den gleichen gemeinsamen globalen Speicher zu, besitzen jedoch auch einen eigenen schnelleren lokalen Speicher (Cache). OpenMP läuft auf genau diesen Systemen.\\
Die Idee hinter OpenMP ist es Schleifen, die normalerweise sequentiell abgearbeitet werden aufzuteilen und von mehreren Prozessoren ausführen zu lassen. Dies ist natürlich nicht für jede Schleife möglich, weshalb der Entwickler entscheiden muss welche Schleifen parallelisiert ausgeführt werden können, welche nicht und welche vielleicht mit Anpassungen parallelisiert werden können. Eine Schleife wird über Pragma-Direktiven für OpenMP markiert und konfiguriert. Ein Beispiel, das zwei Arrays addiert könnte so aussehen:

\begin{lstlisting}[language=C]
#pragma omp parallel for
for (i = 0; i < arrayLength; i++)
    c[i] = a[i] + b[i];
\end{lstlisting}

Dabei wird die Schleife auf alle verfügbaren Prozessoren in einem eigenen Thread aufgeteilt, sodass jeder Prozessor in etwa die gleiche Anzahl an Schleifendurchführungen berechnet. Dies ist offensichtlich nur möglich, wenn die Schleifendurchführungen unabhängig voneinander sind.

\subsection{OpenMPI}
Neben eng gekoppelten Speichersystemen, gibt es auch lose gekoppelte Speichersystem, d.h. es gibt keinen globalen Speicher, sondern jeder Prozessor hat nur seinen eigenen Speicher. Eine Kommunikation erfolgt hier über das Netzwerk. Solche Installationen findet man z.B. in Grids, wo mehrere Rechner wie ein Rechner agieren. Für solche Systeme wurde OpenMPI entwickelt, obwohl es auch auf eng gekoppelten Systemen läuft, indem es eine lose Gekoppeltes simuliert.\\
Der Kern von OpenMPI ist das Austauschen von Informationen zwischen den Prozessoren über "Nachrichten". In der Regel wird die Anwendung von einem Master- oder Root-Prozess initialisert, welcher dann die Daten, die parallelisiert verarbeitet werden sollen an die zur Verfügung stehenden Worker-Prozesse verteilt und nach der Berechnung wieder erhält.\\
%TODO sehr simples Beispiel einfügen?
Um den Nachrichtenaustausch zwischen Prozessen zu in OpenMPI zu realisieren werden die beiden Funktionen Send und Receive benötigt.   
Die Funktion Send ist in der Lage Daten in Form von Variablen von einem Thread zum anderen zu übermitteln.
 
\begin{lstlisting}[language=C]
MPI_Send(void* data,int count,MPI_Datatype datatype,int destination, int tag, MPI_Comm communicator)
\end{lstlisting}

Bei der Verwendung müssen die folgenden Parameter übergeben werden:

\begin{itemize}
\item Zeiger auf die Variable
\item Länge der Variable
\item Datentyp der Variable
\item Zielprozess-ID
\item Tag(optional)
\item Kommunikator
\end{itemize}

Die Funktion Receive hingegen bildet das Gegenstück zu Send, sie ermöglicht das Empfangen von  und Zuordnen von Daten zu einer Variable. Zu beachten ist, dass die Funktion Receive die Eigenschaft hat non-blocking zu sein. Deswegen wird auf eine Nachricht des korrespondierenden  Send solange gewartet bis diese eintrifft.

\begin{lstlisting}[language=C]
MPI_Recv(void* data, int count,MPI_Datatype datatype, int source, int tag, MPI_Comm communicator,MPI_Status* status)
\end{lstlisting}

Bei der Verwendung der Funktion müssen die folgenden Parameter übergeben werden:

\begin{itemize}
\item Zeiger auf die Variable
\item Länge der Variable
\item Datentyp der Variable
\item Quellprozess-ID
\item Tag(optional)
\item Kommunikator
\item Status
\end{itemize}

Für den fehlerfreien Gebrauch von Send und Receive müssen diese zwingend zusammen genutzt werden. Es darf demnach kein Send ohne korrespondierenden Receive existieren, das sonst ein eine interner Fehler von openMPI ausgelöst wird. Gleichfalls darf kein Receive ohne passendes Send vorkommen, andernfalls wird eine Deadlock hervorgerufen, der nicht zur laufzeit auf gelöst werden  kann.

\subsection {False Sharing}
Ein wichtiger Punkt bei der Entwicklung von parallelisiert Anwendungen ist das False-Sharing-Problem, welches dazu führen kann das eine parallelisierte Anwendung langsamer läuft als ihre entsprechende sequentielle Anwendung.\\
Jeder Prozessor hat einen Cache, welcher sehr viele schnellere Zugriffszeiten bietet als der Arbeitsspeicher. Wenn der Prozessor einen Wert einliest, schreibt er diesen zunächst in seinen Cache und arbeitet von dort aus mit diesem. Wenn jetzt ein anderer Prozessor den selben Wert überschreibt, muss der erste Prozessor den aktualisierten Wert wieder aus dem langsameren Arbeitsspeicher laden.\\
In einem Programm in dem abwechselnd zwei Threads immer auf den gleichen Wert schreiben, führt dies dazu, dass der Cache quasi obsolet wird, da jeder Wert aus dem Arbeitsspeicher gelesen werden muss. Das wiederum hat zur Folge, dass das Programm sehr langsam wird.
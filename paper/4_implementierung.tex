% !TEX root = main.tex

\section{Implementierung}
%TODO Text ergänzen
\subsection{Sequentiell}
\begin{lstlisting}[language=C]
for (int i = 1; i < arrLen-1; i++)
	newval[i] = (2 * values[i]) - oldval[i] + c * (values[i-1] - (2 * values[i]) + values[i+1]);

for (int i = 1; i < arrLen-1; i++) 
{
	oldval[i] = values[i];
	values[i] = newval[i];
}
\end{lstlisting}

\begin{lstlisting}[language=C]
for (int i = 1; i < arrLen-1; i++)
{
	if (0 == mode)
		newval[i] = (2 * values[i]) - oldval[i] + c * (values[i-1] - (2 * values[i]) + values[i+1]);
	else if (1 == mode)
		oldval[i] = (2 * newval[i]) - values[i] + c * (newval[i-1] - (2 * newval[i]) + newval[i+1]);
	else
		values[i] = (2 * oldval[i]) - newval[i] + c * (oldval[i-1] - (2 * oldval[i]) + oldval[i+1]);
}
mode++;
if (2 < mode)
	mode = 0;
\end{lstlisting}

\subsection{OpenMP}
\begin{lstlisting}[language=C]
#pragma omp parallel for
for (int i = 1; i < arrLen-1; i++)
[...]

\end{lstlisting}


\subsection{OpenMPI}
%TODO

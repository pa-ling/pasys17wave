\documentclass[12pt,a4paper,ngerman]{article}
\usepackage[utf8]{inputenc}
\usepackage[ngerman]{babel}
\usepackage{csquotes}
\MakeOuterQuote{"}
%
\usepackage{fancyref}
\usepackage{fancyhdr}
\usepackage{xcolor}
\usepackage{url}
\usepackage{makeidx}
\usepackage{listings}

%
% PDF settings
\usepackage[pdftex]{graphicx}
\usepackage[pdfstartview=FitH,pdftitle={Bachelorarbeit},pdfauthor={Patrick Fehling \& Christian Schütt},
colorlinks=false, linktocpage]{hyperref}
\usepackage[a4paper,includeheadfoot,left=2.6cm,right=2.6cm,top=2.54cm,bottom=2.54cm]{geometry}
\usepackage{array}
\setlength\extrarowheight{5pt}

\usepackage[nottoc,numbib]{tocbibind}
\usepackage{hyperref}
\usepackage{caption}

\usepackage{tabularx}
\newcolumntype{L}[1]{>{\raggedright\arraybackslash}p{#1}} % linksbündig mit Breitenangabe
\newcolumntype{C}[1]{>{\centering\arraybackslash}p{#1}} % zentriert mit Breitenangabe
\newcolumntype{R}[1]{>{\raggedleft\arraybackslash}p{#1}} % rechtsbündig mit Breitenangabe

%for e.g. average symbol
\usepackage{amssymb}
\usepackage{float}

%
% Header and Footer Style
\pagestyle{fancy}
\fancyhead{}
\fancyhead[R]{\slshape Patrick Fehling \& Cristian Schütt}
\fancyhead[L]{\slshape\nouppercase{\rightmark}}
\fancyfoot{}
\fancyfoot[C]{\thepage}
\renewcommand{\headrulewidth}{0pt}
\renewcommand{\sectionmark}[1]{\markright{\thesection\ #1}}
\renewcommand{\subsectionmark}[1]{} %Remove \subsection from header
%
% No identation
\setlength\headheight{15pt}
\setlength\parindent{0pt}
%
% Custom commands
\newcommand\zb{z.\,B.\ }
\renewcommand\dh{d.\,h.\ }
\newcommand\parbig{\par\bigskip}
\newcommand\parmed{\par\medskip}
\newcommand{\mailto}[1]{\href{mailto:#1}{#1}}
%
% Java Code Listing Style
\definecolor{darkblue}{rgb}{0,0,.6}
\definecolor{darkgreen}{rgb}{0,0.5,0}
\definecolor{darkred}{rgb}{0.5,0,0}
\lstset{language=Java, basicstyle=\ttfamily\small\upshape, commentstyle=\color{darkgreen}\sffamily,
keywordstyle=\color{darkblue}\rmfamily\bfseries, breaklines=true,tabsize=2,xleftmargin=3mm,
xrightmargin=3mm,numbers=none,frame=none,stringstyle=\color{darkred}, showstringspaces=false}

\begin{document}

\begin{titlepage}
\thispagestyle{empty}

\begin{center}
	\includegraphics[width=0.6\textwidth]{pictures/HTW_Logo}
	
	\vspace{2cm}
	
	\Huge 
	Parallelisierte Berechnung einer Wellenanimation mit OpenMP und OpemMPI
	
	\vspace{2cm}
	\large
	Projekt im Fach "Parallel Systems"
	
	\vspace{2cm}
	
	Prüfer: Sebastian Bauer
	
	\vspace{0.5cm}
	
	Eingereicht von Patrick Fehling und Christian Schütt
	
	\vspace{0.5cm}
	
	31.07.2017
\end{center}


\end{titlepage}

\pagestyle{empty}

\tableofcontents
\setcounter{page}{0}

\clearpage\pagenumbering{arabic}
\pagestyle{fancy}


\section{Einleitung}
\subsection{Projektzielstellung}
\subsection{Aufbau der Arbeit}

\section{Grundlagen}
\subsection{Parallelisierung}
\subsection{OpenMP}
\subsection{OpenMPI}

\section{Analyse}
%wellengleichung
%Mathematikzeug

\section{Entwurf}
%zwei Programme --> warum?
%wie wird seriell ausgeführt?
%Diagramm des Programms ("Klassendiagramm")

\section{Implementierung}
%relevante Code-Snippets

\section{Test}
%manuelle Tests?
%Testskripte

\section{Evaluation}
%Vergleich von seriell und parallel

\section{Fazit}
%Parallelisierung brachte wenig Erfolg

\section{Anhang}
%Verweis auf repository?

\end{document}
